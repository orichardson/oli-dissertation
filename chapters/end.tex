
\section{Conclusions}





\subsection{Sewing Things Back Together, with Self-Consistency}

So far we have been working with the degree of inconsistency of a PDG, a number typically in $[0,\infty]$. 
In \cref{chap:conf}, we saw that the confidence domain 
% $[0,\infty]$ 
$[-\infty, 0]$ is isomorphic to $[0,1]$.
    % and the set of such isomorphisms itself is isomorphic to $(0,\infty)$. 
% For $k > 0$, 
Define the $k$-\emph{self-consistency} of a PDG to be
\[
    \Xi(\dg M) := \mathfrak\exp( -  \aar{\dg M}_\gamma). 
\]


We can summarize and reframe a number of the properties we have seen in terms of
    of the self-consistency $\Xi[\dg M]$
% \begin{prop}
\begin{enumerate}
    \item If $U$ is an event, then $\Xi[U] = {\singlespacing\begin{cases}
        0 & \text{ if } U = \emptyset \\
        1 & \text{ if } U \ne \emptyset
    \end{cases}}$

    \item If $U_1$ and $U_2$ are both events regarding the same set of variables, then 
        $\Xi[U_1 + U_2] = \mathbbm 1[ U_1 \cap U_2 \ne \emptyset]$.  

    \item If $\mu$ is a probability distribution, and $U$ is an event, 
        then $\Xi[\mu + U] = \mu(U)$.
    
    \item If $\Plaus$ is a Dempster-Shafer probability measure, and $U$ is an event, then $\Xi[\Plaus + U] = \Plaus(U)$. 

    % \item If $\dg M$ represents a deterministic classifier and gold standard labels $\Xi(\dg M)$ is the accuracy of the classifier.
    \item If $D = \{(x_i, y_i) \}$ is a set of training data and $f : X \to Y$ is a
        deterministic classifier, 
        then 
        $\Xi[\Pr_D + f ]$ is the accuracy of $f$ on $D$. 

    % If $p(Y|X)$ is instead a probabilistic classifier, then 
    % $\Xi[ \Pr_D !  + p(Y|X)]$
    % is the cross entropy 
    % $\Ex_{D}[ \log \frac1{p(Y|X)} ]$.
    
\end{enumerate}
% \end{prop}


\section{Open Questions}

\paragraph{Structural Information Deficiency and QIM-Compatibility.}
\Cref{sec:qim-info} 

\paragraph{Cpaturing Other Representations.}
In \cref{part:univ-model}, we saw that PDGs and QIM-compatibility,


There's no reason to think that this is 
Object graphs and databases [Spivak].

In \cref{chap:PDG-cat}, we found that PDGs can be viewed as diagrams in a certain category. We saw that categorical limits of PDGs are important. But what about colimits, and natural transformations between them? 
The categorical picture opens a vast landscape of abstract research directions, ranging from variants of PDGs where probability is replaced by a different concept, to connecting the standard defences of information loss \citep[Theorem ??]{leinster2021entropy}.


\section{Applying the Theory}


Conspicuously absent are real-world problems 
\url{https://github.com/orichardson/pdg}.
