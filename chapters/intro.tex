% What is the problem I'm solving?

\section{}



\section{Overview of Results}
\section{Themes and Motifs}

\subsection{Equivalent Representations}

Suppose I learn that 


\subsection{Epistemic Humility}
I distinguish between 

\subsection{A Division between Quantitative and Qualitative}

\subsection{Not Nouns, but Verbs}

I grew up programming in Java. I loved the \href{https://steve-yegge.blogspot.com/2006/03/execution-in-kingdom-of-nouns.html}{kingdom of nouns}.

When I started to learn about the foundations of mathematics, 
I loved 

AI courses where about nouns. 

Then, in 20

\[
\text{Set Theory} : \text{Category Theory}
~~::~~
\text{GOFAI} : \text{ML Systems}
\]



\subsection{Mathematical Precision for Informal Reasoning}
People are often fast and loose with their notation and their math, especially when
    talking about complicated topics with deep foundations.
    
This is a sign that the formalism could be improved.
The whole point of formalism is to support reasoning and thought, 
    in a way that aids precision; if it is too cumbersome to use properly,
    perhaps we would be better off with another formalism.
    
    Either it is too cumbersome to spea
