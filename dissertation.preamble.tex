
%%%%% BIBLIO AND STRUCTURE
\usepackage[numbers]{natbib}
\usepackage{appendix}
% for sub-appendices
% https://tex.stackexchange.com/questions/120716/appendix-after-each-chapter
\AtBeginEnvironment{subappendices}{
	\chapter*{Appendices for Chapter \thechapter}
	% \addcontentsline{toc}{chapter}{Appendices}
	% \addtocontents{toc}{\rule[.1\normalbaselineskip]{1.25in}{.15mm}\par}
	% \addtocontents{toc}{APPENDICES}
	\addtocontents{toc}{\medskip\relax
	 	\rule[0.5ex]{1in}{.15mm}\hfill
		Chapter \thechapter\ Appendices
		\hfill \rule[0.5ex]{1in}{.15mm}
		\hfill\phantom{-}
	}
	\counterwithin{figure}{section}
	\counterwithin{table}{section}
	}
\AtEndEnvironment{subappendices}{%
	\counterwithin{figure}{chapter}
	\counterwithin{table}{chapter}
	% \counterwithout{figure}{section}
	% \counterwithout{table}{section}
	}


%%%%% EXTRA COMMANDS TO HIDE STUFF;
% ideally, it will be possible to delete all of this at the end.
% \newcommand{\notation}[1]{\ignorespaces} % disable notation
\newcommand{\notation}[1]{{\color{red}#1}~~} % enable notation
\newcommand{\vfull}[1]{{\color{orange}#1}} % previously cut "full-version"; highlight
\newcommand\fadeout[1]{{\color{gray!50}\small#1}}
\newcommand\vjoe[1]{{\color{red!80!black}#1}}
\newcommand\voli[1]{{\color{orange!80!black}#1}}
\newcommand\commentout[1]{}
\newcommand\discard[1]{{\color{orange!80!black}$\blacksquare$}}
\newcommand{\TODO}[1][INCOMPLETE]{{\centering\Large\color{red}$\langle$~\texttt{#1}~$\rangle$\par}}
\newcommand\hidewithbox[1]{{\color{red}$\blacksquare$}}

%%%%% STANDARD PACKAGES
\usepackage{mathtools}		%also loads amsmath
\usepackage{faktor}
\usepackage{amssymb,bbm}
\usepackage{booktabs}
\usepackage{cancel}
	\newcommand{\Cancel}[2][black]{{\color{#1}\cancel{\color{black}#2}}}
\usepackage{nicefrac}
\usepackage[noend]{algorithmic}
\usepackage{algorithm}
% \usepackage{algpseudocode}
\usepackage[dvipsnames, cmyk]{xcolor} % must be loaded before tikz
\usepackage{tikz} % setup comes later
\usepackage{enumitem}
\usepackage{array}
\newcolumntype{L}{>{$}l<{$}} % switch mathmode, a description column
% \usepackage{trimclip}
\usepackage{xstring}
\usepackage{environ}% http://ctan.org/pkg/environ;
	% required for use of label_matrix.tex.
\usepackage{nameref,hyperref}%\Autoref is defined by thmtools
\usepackage[noabbrev,nameinlink,capitalize]{cleveref}% n.b. cleveref after! hyperref

%%%%% visual appearance
\colorlet{color1}{blue}
\colorlet{color2}{color1>wheel,1,3}
\colorlet{color3}{color1>wheel,2,3}
\definecolor{deepgreen}{rgb}{0,0.5,0}
\colorlet{pinkish}{color3!25!magenta}
\usepackage{relsize}
\usepackage{wrapfig}
\usepackage{scalerel}
% \hypersetup{colorlinks=true, linkcolor=blue, urlcolor=magenta, citecolor=deepgreen}
\hypersetup{colorlinks=true, linkcolor=color1, urlcolor=magenta, citecolor=color2}
% \usepackage{parskip}
\usepackage{microtype}      % microtypography
% \usepackage[format=plain,labelfont={sl},textfont={it,small},margin=1cm]{caption}
\usepackage[format=hang,labelfont={it},textfont={small}]{caption}
\usepackage{subcaption}
%\usepackage{subfig}
%\captionsetup[subfigure]{subrefformat=simple,labelformat=simple}
%\renewcommand\thesubfigure{(\alph{subfigure})}


%%%%% PREPARING EXTRA SYMBOLS
\usepackage{stmaryrd} % disable me for Joe
\SetSymbolFont{stmry}{bold}{U}{stmry}{m}{n}
\makeatletter
% load lBrace, rBrace from stix... but don't load stix
\@ifpackageloaded{stix}{}{
	\DeclareFontEncoding{LS2}{}{\noaccents@} \DeclareFontSubstitution{LS2}{stix}{m}{n}
	\DeclareSymbolFont{stix@largesymbols}{LS2}{stixex}{m}{n}
	\SetSymbolFont{stix@largesymbols}{bold}{LS2}{stixex}{b}{n}
	\DeclareMathDelimiter{\lBrace}{\mathopen} {stix@largesymbols}{"E8}{stix@largesymbols}{"0E}
	\DeclareMathDelimiter{\rBrace}{\mathclose}{stix@largesymbols}{"E9}{stix@largesymbols}{"0F}
}
% who doesn't love dutchcal?
\DeclareMathAlphabet{\mathdcal}{U}{dutchcal}{m}{n}
\DeclareMathAlphabet{\mathbdcal}{U}{dutchcal}{b}{n}
\DeclareRobustCommand{\shortto}{\mathrel{\mathpalette\short@to\relax}}
\newcommand{\short@to}[2]{%
	\mkern2mu
	\clipbox{{.5\width} 0 0 0}{$\m@th#1\vphantom{+}{\shortrightarrow}$}%
}
% better-spaced cases environment
\makeatletter
\renewenvironment{cases}[1][l]{\matrix@check\cases\env@cases{#1}}{\endarray\right.}
\def\env@cases#1{%
	\let\@ifnextchar\new@ifnextchar
	\left\lbrace\def\arraystretch{1.2}%
	\array{@{}#1@{\quad}l@{}}}
\makeatother
\newcommand\numberthis{\addtocounter{equation}{1}\tag{\theequation}}
\makeatother
%% don't double space matrices please...
\makeatletter
\def\env@matrix{\hskip -\arraycolsep
  \let\@ifnextchar\new@ifnextchar
  \linespread{1}\selectfont
  \renewcommand{\arraystretch}{1.2}%
  \array{*\c@MaxMatrixCols c}}
\makeatother


%%% GENERAL SYMBOLS 
\let\Horig\H
\let\H\relax
\let\nf\nicefrac
\DeclareMathOperator{\H}{\mathrm{H}} % Entropy
\DeclareMathOperator{\I}{\mathrm{I}} % Information
\DeclareMathOperator*{\Ex}{\mathbb{E}} % Expectation
\DeclareMathOperator*{\argmin}{arg\,min} % argmin
\DeclareMathOperator*{\argmax}{arg\,max} % argmax
% \DeclareMathOperator*{\argmin}{\arg\!\min}
% \DeclareMathOperator*{\argmax}{\arg\!\max}
\DeclareMathOperator{\im}{\mathrm{im}} % image of a map
\DeclareMathSymbol{\shortminus}{\mathbin}{AMSa}{"39} %shorter minus
\DeclarePairedDelimiter{\norm}{\Vert}{\Vert} % paired norm
\DeclarePairedDelimiter{\pqty}{\lparen}{\rparen} % paired parens
\newcommand{\ifrac}[2]{{#1}/{#2}} % inline fraction, (I guess?)
\DeclareMathOperator{\Supp}{\mathrm{Supp}} %support
\newcommand{\Unif}{\mathrm{Unif}} % uniform distribution
	
\newcommand{\Rext}{\mskip1mu\overline{\mskip-1mu\mathbb R\!}\,}
\newcommand\mat[1]{\mathbf #1} % short matrix / \mathbf 
\newcommand{\indi}[1] % indicator function
	{\mathbbm{1}_{\left[\vphantom{\big[}#1 \vphantom{\big]}\right]}} 
	
\DeclarePairedDelimiterX{\infdivx}[2]{(}{)}{#1\;\delimsize\|\;#2}
\newcommand{\thickD}{I\mkern-8muD}
\newcommand{\kldiv}{\thickD\infdivx} % D_\mathrm{KL}
% graphical models stuff
\newcommand{\CI}{\mathrel{\perp\mspace{-10mu}\perp}} % independence 
\newcommand{\Pa}{\mathop{\mathbf{Pa}}}
\newcommand{\tto}{\rightarrow\mathrel{\mspace{-15mu}}\rightarrow}
% Other notions of uncertainty
\newcommand\Bel{\mathit{Bel}}
\newcommand\Plaus{\mathit{Plaus}}

\newcommand\At[1]{\mathit{A\mkern-2mu t}(#1)} % boolean algebra atoms
\newcommand\lang[1]{\mathcal L^{\mathit{#1}}} % languages
\newcommand\geqc{\succcurlyeq}
\newcommand\leqc{\preccurlyeq}


% (these are all defined in pdg-preamble.tex)
% \DeclarePairedDelimiter{\cbbr}{\lBrace}{\rBrace}
% \DeclarePairedDelimiter{\bbr}{\llbracket}{\rrbracket}
% \DeclarePairedDelimiter{\ppr}{\llparenthesis}{\rrparenthesis}
% \DeclarePairedDelimiterX{\aar}[1]{\langle}{\rangle}
% 	{\mspace{3mu}\mathllap{\delimsize\langle}#1\mathrlap{\delimsize\rangle}\mspace{3mu}}
% \DeclarePairedDelimiterXPP{\aard}[1]{}{\langle}{\rangle}{_{\!_\downarrow}}
% 	{\mspace{-3.5mu}\delimsize\langle#1\delimsize\rangle\mspace{-3.5mu}}
% 


\relax%%%%% STANDARD SETS / SPACES / CATEGORIES %%%%
    % \newcommand{\Stoch}{\mathbf{Stoch}}
    \newcommand{\Stoch}{\mathbb{S}\mathbf{toch}}
    % \newcommand{\FinSet}{\mathbf{FinSet}}
    % \newcommand{\Set}{\mathbf{Set}}
    \newcommand\CMon{\mathbb{C\mskip-1.5muM}\mskip-1.3mu\mathbf{o\mskip-1mun}}
    \newcommand\Mon{\mathbb{M}\mskip-1.5mu\mathbf{o\mskip-1mun}}
    \newcommand\Set{\ensuremath{\mathbb{S}\mathbf{et}}}
    \newcommand\Top{\ensuremath{\mathbb{T}\mathbf{op}}}
    \newcommand\Rel{\ensuremath{\mathbb{R}\mathbf{el}}}
    \newcommand\Diff{\ensuremath{\mathbb{D}\mathbf{iff}}}
    \newcommand\Meas{\ensuremath{\mathbb{M}\mathbf{eas}}}
    \newcommand\FinSet{\ensuremath{\mathbb{F}\mathbf{in}\mathbf{S}\mathbf{et}}}
	\newcommand\Two{\mathbbm 2}
	\newcommand\PDG{\mathbb{P}\mathrm{DG}}
    \newcommand\Semiringoid{\mathbb S\mathbf{emiringoid}}
    \newcommand\Mat{\mathop{\mathrm{Mat}}}
    
\relax %%% category theory macros %%%
    %\newcommand\CMon{\mathbb{C}\!\mathrm{o}\mathbb{M}} \newcommand\ICMon{\mathbb{I\mskip-1muC\mskip-1.5muM}\mskip-1.3mu\mathbf{o\mskip-1mun}}
    \DeclareMathOperator{\ob}{\mathop{\mathrm{ob}}}
    \DeclareMathOperator{\id}{\mathrm{id}}
    \DeclareMathOperator{\Hom}{\mathop{\mathrm{Hom}}}

	
	
%%%%%  SPECIAL COMMANDS FOR PDG-AAAI  %%%%%
\newcommand{\Gr}{\mathcal G}
\newcommand{\alle}[1][a]{_{\ed {#1}XY}}
% \newcommand\GFE{\mathit{G\mkern-4mu F\mkern-4.5mu E}}
\newcommand\VFE{\mathit{V\mkern-4mu F\mkern-4.5mu E}}
% conversions
\newcommand{\WFGof}[1]{\Psi_{{#1}}}
\newcommand{\FGof}[1]{\Phi_{{#1}}}


%%%%%  SPECIAL COMMANDS FOR QIM  %%%%%%%
\newcommand{\derind}[2]{ {#1}^{+\tilde{#2}(\X^{\U})}}
\newcommand\U{\mathcal U}
\newcommand\enV{\mathcal V} % for endogenous variables

%%%%%  SPECIAL COMMANDS FOR ONE-TRUE-LOSS %%%%%%%
\newcommand{\datadist}[1]{\Pr\nolimits_{#1}}
		% \newcommand{\datadist}[1]{p_\text{data}}


%%%%%  SPECIAL COMMANDS FOR INFERENCE  %%%%%%%
\newcommand\bmu{\boldsymbol\mu}
\newcommand\C{\mathdcal C}
\newcommand\minimize{\mathop{\scalebox{0.98}{$\mathsf{minimize}$}}\limits}
\newcommand\maximize{\mathop{\scalebox{0.98}{$\mathsf{maximize}$}}\limits}
\newcommand\subjto{{\scalebox{0.98}{$\mathsf{subject~to}$}}}

\newcommand\actree{tree marginal}
\newcommand\AcTree{Tree Marginal}
\newcommand\cactree{calibrated \actree}

		
%%%%%  SPECIAL COMMANDS FOR LIR %%%%%%%%%
\newcommand\Ctx{\dg{C\mskip-2mut\mskip-2mux}}
\newcommand\Mm{\dg{M}}
\newcommand\MThetadense{\dg{M}(\mskip-1mu\Theta\mskip-1mu)}
\newcommand\Attn{\dg{A\mskip-2.2mut\mskip-2mut\mskip-2mun}}
\newcommand\Ctrl{\dg{C\mskip-2mut\mskip-2mur\mskip-2mul}}

\newcommand\attn{A}
\newcommand\ctrl{C}

\relax % Short arrows.
    \newcommand{\veryshortarrow}[1][3pt]{\mathrel{%
	    \vcenter{\hbox{\rule[-.5\fontdimen8\textfont3]{#1}{\fontdimen8\textfont3}}}%
	    \mkern-4mu\hbox{\usefont{U}{lasy}{m}{n}\symbol{41}}}}
    % \newcommand\sto{{{\scalebox{1}[1]{$\scriptstyle\to$}}}}
    %
    \newsavebox{\stosbox}
    \sbox{\stosbox}{\begin{tikzpicture}\draw[->] (0,0) to (0.15,0);\end{tikzpicture}}
    \newcommand\sto{\usebox\stosbox}
    %%%
    %%% hacky: now I have to do the same thing for green and blue
    \newsavebox{\stosboxgreen}
    \sbox{\stosboxgreen}{\begin{tikzpicture}\draw[->,green!70!black] (0,0) to (0.15,0);\end{tikzpicture}}
    \newcommand\stogreen{\usebox\stosboxgreen}
    \newsavebox{\stosboxblue}
    \sbox{\stosboxblue}{\begin{tikzpicture}\draw[->,blue] (0,0) to (0.15,0);\end{tikzpicture}}
    \newcommand\stoblue{\usebox\stosboxblue}

%%%%%%% SPECIAL COMMANDS FOR REASONING WITH PDGS %%%%%%
\newcommand\False{\dg{F\mskip-5mua\!l\mskip-4mus\mskip-1.5mue}}

%%%%%  SPECIAL COMMANDS FOR CONFIDENCE  %%%%%
\newcommand\confdom{[\bot,\!\top]}
\newcommand\Lrn{\mathit{Lrn}}
\newcommand{\ext}[1]{\overline #1} %  extension of \Phi and F to mixture.
\usepackage{xparse}
\let\realItem\item % save a copy of the original item; will be temporarily swapped out in envirnoments below.
\makeatletter
\NewDocumentCommand\myItemboldperiod{o}{%
   \IfNoValueTF{#1}%
      {\realItem}% add an item
      {\realItem[\conflabel{#1}]%
        \def\@currentlabel{#1}%
        \protected@edef\cref@currentlabel{[CFaxiomsi][][]#1}% \label[#1]{#2}
        }% add an item and update label
}
\makeatother
% \newcommand\conflabel[1]{\textbf{#1.}}
\newcommand\conflabel[1]{\textbf{[#1]}}
\newlist{LrnAxioms}{enumerate}{1}
	\setlist[LrnAxioms]{%
	    resume,%
	    label=\conflabel{L\arabic{*}},
	    ref={L\arabic*},
	    leftmargin=*,
	    itemindent=1.5em, labelsep=1em, topsep=1ex,
	    before=\let\item\myItemboldperiod,
    }
	\crefname{LrnAxiomsi}{}{}
	\crefrangeformat{LrnAxiomsi}{#3#1#4#5--\crefstripprefix{#1}{#2}#6}
\newlist{LrnBelAxioms}{enumerate}{1}
	\setlist[LrnBelAxioms]{
	    resume,%
	    label=\conflabel{LB\arabic{*}},
	    ref={LB\arabic*},
	    leftmargin=*,
	    itemindent=1.5em, labelsep=1em, topsep=1ex,
	    before=\let\item\myItemboldperiod,
    }
	\crefname{LrnBelAxiomsi}{}{}
	\crefrangeformat{LrnBelAxiomsi}{#3#1#4#5--\crefstripprefix{#1}{#2}#6}
 

%%%%%  DATABASE SYMBOLS  %%%%%
\newcommand{\D}{\mathbdcal D} % for a database
\newcommand{\Attrs}{\mathdcal A}
\newcommand{\Idx}{\mathcal J}
\newcommand{\Doms}{{\mathcal D}}
% \newcommand{\Rels}{{\mathcal R}}
\newcommand{\Rels}{{\mathbf R}}
\newcommand{\Cols}{\mathcal C}%{\sigma}
\newcommand{\Sch}{\mathdcal S}
\newcommand{\Keys}{\mathcal K}
\newcommand{\DBProb}{\mathdcal P}
\newcommand{\arity}{\mathit{ar}}

\newcommand\Varis{\V\kern-.65pt\mathit{ars}}%
\newcommand\Bmid{\mathrel{\Big|}}%
\newcommand\vals{\mathbf{vals}}
\newcommand\nj{\bowtie}






%%%%%%%%%%%%%%%%%%%%%%%%%%%%%%%%%%%%%%%%%%%%%%%%%%%%%%%%%%

% before theorems I need this:
\def\structurelevels{part,section,subsection,subsubsection,paragraph}%
\newcommand\nextstructurelevel{%
	\ifnum\value{section}=0 section\else%
	  \ifnum\value{subsection}=0 subsection\else%
	    \ifnum\value{subsubsection}=0 subsubsection\else%
		paragraph\fi\fi\fi}

%% The loop didn't work :( burrying it for now...
% \expandafter\newcommand\nextstructurelevel{%
% 	{\foreach\strl in \structurelevels{%
% 		\ifnum\expandafter\value{\strl}=0%
% 			\strl\breakforeach\fi%
% 		}%
% 	}}
% \newcommand\nextstructurelevel{subsubsection}

%%%%%%%%%%%%%%%% THEOREMS %%%%%%%%%%%%%%%%%%%%%%%
\usepackage{amsthm}
\usepackage{thmtools} % asmsymb must be loaded also.
\usepackage[framemethod=TikZ]{mdframed}
\usepackage[nobox]{restatelinks}

	\makeatletter\begingroup

	\@for\theoremstyle:=definition,remark,plain\do{%
			\expandafter\g@addto@macro\csname th@\theoremstyle\endcsname{%
				\addtolength\thm@preskip\parskip}
			}\endgroup
	% add theorems, examples,etc to table of contents
	\def\thmTOCadd{
		\edef\asdffdsa{\nextstructurelevel}
		% \edef\asdffdsa{subsubsection}
		\ifx\thmt@optarg\@empty
			\addcontentsline{toc}{\asdffdsa}{%
			\texorpdfstring{
			 	\makebox[3cm][r]{\color{gray}{\it\thmt@thmname}~%
				\csname the\thmt@envname\endcsname}}%
			{\thmt@thmname\csname the\thmt@envname\endcsname: \thmt@optarg}}
		\else\addcontentsline{toc}{\asdffdsa}{%
			\texorpdfstring{
			 	\makebox[3cm][r]{\color{gray}{\it\thmt@thmname}~%
				\csname the\thmt@envname\endcsname:}\hspace{1em} {\color{black!50!gray}\small\thmt@optarg}}%
			{\thmt@thmname\csname the\thmt@envname\endcsname: \thmt@optarg} %
		}\fi}
	%... except actually we don't want to add this to TOC, I think...
	\def\thmTOCadd{}
	
	% macro for clearing theorems (if preamble adapted)
	% my answer presented here: https://tex.stackexchange.com/a/296184/191268
	\def\cleartheorem#1{%
		\expandafter\let\csname#1\endcsname\relax
		\expandafter\let\csname c@#1\endcsname\relax
	}
	\def\clearthms#1{ \@for\tname:=#1\do{\cleartheorem\tname} }
	\makeatother

	\theoremstyle{plain}
	\declaretheorem[within=section,name=Theorem,postheadhook={\thmTOCadd}]{theorem}
	\newtheorem{coro}{Corollary}[theorem]
	\declaretheorem[sibling=theorem,name=Proposition, postheadhook={\thmTOCadd}]{prop}
	\newtheorem{lemma}[theorem]{Lemma}
	\newtheorem{fact}[theorem]{Fact}
    \newtheorem{falsity}[theorem]{Falsity}
	\newtheorem{conj}[theorem]{Conjecture}
    \newtheorem{claim}{Claim}
	\declaretheorem[numberwithin=theorem,name=Claim]{iclaim} % internal claims for lproofs.
	
	\theoremstyle{definition}
	\newtheorem*{defn*}{Definition}
	\declaretheorem[name=Definition,style=definition,qed=$\square$,within=section, postheadhook={\thmTOCadd}]{defn}
	\declaretheorem[name=Construction,qed=$\square$,sibling=defn, postheadhook={\thmTOCadd}]{constr}
	% \declaretheorem[qed=$\square$]{example}
	\declaretheorem[name=Example,qed=$\triangle$,postheadhook={\thmTOCadd}]{example}

	\theoremstyle{remark}
	\newtheorem*{remark}{Remark}

	
\crefname{example}{Example}{Examples}
\crefname{defn}{Definition}{Definitions}
\crefname{prop}{Proposition}{Propositions}
\crefname{constr}{Construction}{Constructions}
\crefname{conj}{Conjecture}{Conjectures}
\crefname{fact}{Fact}{Facts}
\crefname{claim}{Claim}{Claims}
\crefname{iclaim}{Claim}{Claims}
\crefname{coro}{Corolary}{Corolaries}


%% ANNOTATION
\newcommand{\todo}[1]{{\color{red}\ \!\Large\smash{\textbf{[}}{\normalsize\textsc{todo:} #1}\ \!\smash{\textbf{]}}}}
\newcommand{\note}[1]{{\color{blue}\ \!\Large\smash{\textbf{[}}{\normalsize\textsc{note:} #1}\ \!\smash{\textbf{]}}}}
\newcommand{\moveme}[1]{{\color{purple}\ \!\Large\smash{\textbf{[}}{\normalsize\textsc{moveme:} #1}\ \!\smash{\textbf{]}}}}

\mdfdefinestyle{mybox}{skipabove=1em,skipbelow=1em,roundcorner=2pt}
\newmdenv[style=mybox,fontcolor=black!25, backgroundcolor=black!3,linecolor=black!20]{inactive}
\newmdenv[style=mybox,backgroundcolor=color3!5,linecolor=color3!20]{highlight-changes}
\newmdenv[style=mybox,linecolor=color2!50,topline=false,bottomline=false,rightline=false,innerleftmargin=1em,linewidth=5pt]{leftbar}
\newmdenv[roundcorner=5pt, subtitlebelowline=true,subtitleaboveline=true, subtitlebackgroundcolor=color1!70!white,
 	backgroundcolor=color3!20!white, frametitle={Annotating},frametitlerule=true, frametitlebackgroundcolor=color1!70!white, skipabove=1em,skipbelow=1em,]{annotating}
\newmdenv[roundcorner=5pt, backgroundcolor=pinkish!20!white, frametitle={$\langle$under construction $\rangle$},frametitlerule=false,
innertopmargin=2pt, frametitlebelowskip=3pt, frametitleaboveskip=2pt, frametitlebackgroundcolor=pinkish!70!white, skipabove=1em,skipbelow=1em, frametitlefont={\normalfont\itshape},leftmargin=-10pt, rightmargin=-10pt]
		{wip}
% \usetikzlibrary{external}
% \tikzexternalize[prefix=tikz/]  % activate!
% \usepackage{etoolbox}
% \AtBeginEnvironment{tikzcd}{\tikzexternaldisable} %... except careful of tikzcd...
% \AtEndEnvironment{tikzcd}{\tikzexternalenable}


% %%%%%%%%%%%%%%%% TIKZ SETUP %%%%%%%%%%%%%%%%%%%%%
% 	\usetikzlibrary{positioning,fit,calc, decorations, arrows, shapes, shapes.geometric}
% 	\usetikzlibrary{patterns,backgrounds}
% 	\usetikzlibrary{cd}
% 
% 	\pgfdeclaredecoration{arrows}{draw}{
% 		\state{draw}[width=\pgfdecoratedinputsegmentlength]{%
% 			\path [every arrow subpath/.try] \pgfextra{%
% 				\pgfpathmoveto{\pgfpointdecoratedinputsegmentfirst}%
% 				\pgfpathlineto{\pgfpointdecoratedinputsegmentlast}%
% 			};
% 	}}
% 	%%%%%%%%%%%%
% 	\tikzset{AmpRep/.style={ampersand replacement=\&}}
% 	\tikzset{center base/.style={baseline={([yshift=-.8ex]current bounding box.center)}}}
	\tikzset{paperfig/.style={center base,scale=1.0, every node/.style={transform shape}}}
	% \tikzset{paperfig/.style={center base,scale=0.9, every node/.style={transform shape}}}
% 
% 	\tikzset{is bn/.style={background rectangle/.style={fill=blue!35,opacity=0.3, rounded corners=5},show background rectangle}}
% 	% Node Stylings
% 	\tikzset{dpadded/.style={rounded corners=2, inner sep=0.7em, draw, outer sep=0.3em, fill={black!50}, fill opacity=0.08, text opacity=1}}
% 	% \tikzset{active/.style={fill=blue, fill opacity=0.1}}
% 	% \tikzset{square/.style={regular polygon,regular polygon sides=4, rounded corners = 0}}
% 	% \tikzset{octagon/.style={regular polygon,regular polygon sides=8, rounded corners = 0}}
% 	\tikzset{dpad0/.style={outer sep=0.05em, inner sep=0.3em, draw=gray!75, rounded corners=4, fill=black!08, fill opacity=1}}
% 	\tikzset{dpad1/.style={outer sep=0.1em, inner sep=0.4em, draw=gray!75, rounded corners=3, fill=black!08, fill opacity=1}}
% 	\tikzset{dpad/.style args={#1}{every matrix/.append style={nodes={dpadded, #1}}}}
% 	\tikzset{light pad/.style={outer sep=0.2em, inner sep=0.5em, draw=gray!50}}
% 
% 	\tikzset{arr/.style={draw, ->, thick, shorten <=3pt, shorten >=3pt}}
% 	\tikzset{arr0/.style={draw, ->, thick, shorten <=0pt, shorten >=0pt}}
% 	\tikzset{arr1/.style={draw, ->, thick, shorten <=1pt, shorten >=1pt}}
% 	\tikzset{arr2/.style={draw, ->, thick, shorten <=2pt, shorten >=2pt}}
% 	\tikzset{archain/.style args={#1}{arr, every arrow subpath/.style={draw,arr, #1}, decoration=arrows, decorate}}
% 
% 
	\usetikzlibrary{decorations.pathmorphing}
	\tikzset{fgnode/.style={dpadded,inner sep=0.6em, circle},
	factor/.style={light pad, fill=black}}
% 
% 	% For illustrating what's inside of nodes:
% 	\tikzset{alternative/.style args={#1|#2|#3}{name=#1, circle, fill, inner sep=1pt,label={[name={lab-#1},gray!30!black]#3:\scriptsize #2}} }
% 
% 	\tikzset{bpt/.style args={#1|#2}{alternative={#1|#2|above}} }
% 	\tikzset{tpt/.style args={#1|#2}{alternative={#1|#2|below}} }
% 	\tikzset{lpt/.style args={#1|#2}{alternative={#1|#2|left}} }
% 	\tikzset{rpt/.style args={#1|#2}{alternative={#1|#2|right}} }
% 	\tikzset{pt/.style args={#1}{alternative={#1|#1|above}} }
% 	\tikzset{mpt/.style args={#1|#2}{name=#1, circle, fill, inner sep=1pt,label={[name={lab-#1},gray]\scriptsize #2}} }
% 
% 	\tikzset{Dom/.style args={#1 (#2) around #3}{dpadded, name=#2, label={[name={lab-#2},align=center] #1}, fit={ #3 } }}
% 	\tikzset{bDom/.style args={#1 (#2) around #3}{dpadded, name=#2, label={[name={lab-#2},align=center]below:#1}, fit={ #3 } }}
% 
% 
% 	\newcommand\cmergearr[4]{
% 		\draw[arr,-] (#1) -- (#4) -- (#2);
% 		\draw[arr, shorten <=0] (#4) -- (#3);
% 	}
% 	\newcommand\mergearr[3]{
% 		\coordinate (center-#1#2#3) at (barycentric cs:#1=1,#2=1,#3=1.2);
% 		\cmergearr{#1}{#2}{#3}{center-#1#2#3}
% 	}
% 	\newcommand\cunmergearr[4]{
% 		\draw[arr,-, , shorten >=0] (#1) -- (#4);
% 		\draw[arr, shorten <=0] (#4) -- (#2);
% 		\draw[arr, shorten <=0] (#4) -- (#3);
% 	}
% 	\newcommand\unmergearr[3]{
% 		\coordinate (center-#1#2#3) at (barycentric cs:#1=1.2,#2=1,#3=1);
% 		\cunmergearr{#1}{#2}{#3}{center-#1#2#3}
% 	}
% 
% 
	\usetikzlibrary{matrix}
	\tikzset{toprule/.style={%
	        execute at end cell={%
	            \draw [line cap=rect,#1]
	            (\tikzmatrixname-\the\pgfmatrixcurrentrow-\the\pgfmatrixcurrentcolumn.north west) -- (\tikzmatrixname-\the\pgfmatrixcurrentrow-\the\pgfmatrixcurrentcolumn.north east);%
	        }
	    },
	    bottomrule/.style={%
	        execute at end cell={%
	            \draw [line cap=rect,#1] (\tikzmatrixname-\the\pgfmatrixcurrentrow-\the\pgfmatrixcurrentcolumn.south west) -- (\tikzmatrixname-\the\pgfmatrixcurrentrow-\the\pgfmatrixcurrentcolumn.south east);%
	        }
	    },
	    leftrule/.style={%
	        execute at end cell={%
	            \draw [line cap=rect,#1] (\tikzmatrixname-\the\pgfmatrixcurrentrow-\the\pgfmatrixcurrentcolumn.north west) -- (\tikzmatrixname-\the\pgfmatrixcurrentrow-\the\pgfmatrixcurrentcolumn.south west);%
	        }
	    },
	    rightrule/.style={%
	        execute at end cell={%
	            \draw [line cap=rect,#1] (\tikzmatrixname-\the\pgfmatrixcurrentrow-\the\pgfmatrixcurrentcolumn.north east) -- (\tikzmatrixname-\the\pgfmatrixcurrentrow-\the\pgfmatrixcurrentcolumn.south east);%
	        }
	    },
	    table with head/.style={
		    matrix of nodes,
		    row sep=-\pgflinewidth,
		    column sep=-\pgflinewidth,
		    nodes={rectangle,minimum width=2.5em, outer sep=0pt},
		    row 1/.style={toprule=thick, bottomrule},
  	    }
	}
% 
% 
% 
% % \NewEnviron{ctikzpicture}{\begin{center}\expandafter\begin{tikzpicture}\BODY\end{tikzpicture}\end{center}}
% % \usetikzlibrary{shapes.geometric}
% \usetikzlibrary{backgrounds}
% \tikzset{dpad0/.style={outer sep=0.05em, inner sep=0.3em, draw=gray!75, rounded corners=4, fill=black!08, fill opacity=1}}
% \tikzset{arr0/.style={draw, ->, thick, shorten <=0pt, shorten >=0pt}}
% \tikzset{arr1/.style={draw, ->, thick, shorten <=1pt, shorten >=1pt}}
% \tikzset{arr2/.style={draw, ->, thick, shorten <=2pt, shorten >=2pt}}

\usetikzlibrary{patterns,backgrounds}
\tikzset{is bn/.style={background rectangle/.style={fill=blue!35,opacity=0.3, rounded corners=5},show background rectangle}}
% % \newcommand\lab[1]{(#1)(lab-#1)}
% 
% 
% %%%%%%%%%%%%%% label_matrix.tex %%%%%%%%%%%%%%%%
% % % Wow this works I'm brilliant
\def\wrapwith#1[#2;#3]{
	\expandarg\IfSubStr{#1}{,}{
		\expandafter#2{\expandarg\StrBefore{#1}{,}}
		\expandarg\StrBehind{#1}{,}[\tmp] 
		\xdef\tmp{\expandafter\unexpanded\expandafter{\tmp}}
		#3
		\wrapwith{\tmp}[#2;{#3}]
	}{ \expandafter#2{#1} }
}
\def\hwrapcells#1[#2]{\wrapwith#1[#2;&]}
\def\vwrapcells#1[#2]{\wrapwith#1[#2;\\]}
\NewEnviron{mymathenv}{$\BODY$}

\newcommand{\smalltext}[1]{\text{\footnotesize#1}}
\newsavebox{\idxmatsavebox}
\def\makeinvisibleidxstyle#1#2{\phantom{\hbox{#1#2}}}
\newenvironment{idxmatphant}[4][\color{gray}\smalltext]{%
	\def\idxstyle{#1}
	\def\colitems{#3}
	\def\rowitems{#2}
	\def\phantitems{#4}
	\begin{lrbox}{\idxmatsavebox}$%$\begin{mymathenv}
	\begin{matrix}  \begin{matrix} \hwrapcells{\colitems}[\idxstyle]  \end{matrix} \\[0.1em]
		\left[ 
		\begin{matrix}
			\hwrapcells{\phantitems}[\expandafter\makeinvisibleidxstyle\idxstyle]  \\[-1em]
	}{
		\end{matrix}\right]		&\hspace{-0.5em}\begin{matrix*}[l] \vwrapcells{\rowitems}[\idxstyle] \end{matrix*}
	\end{matrix}%
	$%\end{mymathenv}
	\end{lrbox}%
	\raisebox{0.75em}{\usebox\idxmatsavebox}
%	\vspace{-0.5em}
}

\newenvironment{idxmat}[3][\color{gray}\smalltext]
	{\begingroup\idxmatphant[#1]{#2}{#3}{#3}}
	{\endidxmatphant\endgroup}

\newenvironment{sqidxmat}[2][\color{gray}\smalltext]
	{\begingroup\idxmat[#1]{#2}{#2}}
	{\endidxmat\endgroup}
	
	
%%%%%%%%%%%%
% better alignment for cases
\makeatletter
\renewenvironment{cases}[1][l]{\matrix@check\cases\env@cases{#1}}{\endarray\right.}
\def\env@cases#1{%
	\let\@ifnextchar\new@ifnextchar
	\left\lbrace\def\arraystretch{1.2}%
	\array{@{}#1@{\quad}l@{}}}
\makeatother

