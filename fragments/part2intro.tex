% We now argue that PDGs 

In \cref{sec:defend-inconsist}, we argued that inconsistency is worth representing, 
\newmaterial{%
and perhaps less problematic than it appears.
But that is a long way from an endoresement of inconsistency; holding conflicting beliefs is clearly a bad thing, as pointed 
In fact, inconsistency is universally abhorred
    \citep{sep-dutch-book,descartes,finocchiaro1981fallacies}.  
Yet our shared distaste for inconsistency makes the concept particularly 
    useful, as will 
        % see in \cref{part:univ-objective}.
        soon see.
}%
% this makes the concept particularly useful.

We have already seen PDGs capture a wide variety of epistemic representations used in the litereature.
\newmaterial{%
Because PDGs can express so many different states of mind, it follows that our formal definition of the \emph{inconsistency} \cref{eqn:inconsistency-defn} of a PDG applies in a wide variety of contexts. 

}%
% Here, we argue that minimizing inconsistency is a \emph{universal objective}: it applies in almost any context, and reducing inconsistency is essentially always an appropriate thing to do.

In this part, we will 
% see how much of what we 
make this argument by showing that many computations of interest---%
especially those that form the basis of modern machine learning and AI systems---%
can be viewed as approaches to inconsistency minimization.
Thus, it can be viewed as an empirical argument, showing that, in practice, 
people do want to reduce inconsistency in essentially every context. 
}%
