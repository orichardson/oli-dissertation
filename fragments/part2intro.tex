
\newmaterial{%
% So as not to keep anyone in suspense, the \emph{universal objective} after which this part is named 
So as not to keep anyone in suspense, this \emph{universal objective}
    is to minimize inconsistency.  
    % \newmaterial{%
To a discerning eye, the word \emph{universal} captures four closely related concepts. 

    % }% 
%
From the very beginning (\cref{sec:defend-inconsist,sec:pdg-intro-examples}), we have argued that inconsistency is worth representing, 
    and less problematic than had been previously thought.
But that is a long way from an endorsement of holding inconsistent beliefs.
%
It seems everyone agrees that conflicting beliefs is worthy of derision,
    a position which has also been argued by scholars from many different angles \citep{sep-dutch-book,descartes,finocchiaro1981fallacies,priest1996paraconsistent}. 
    % In fact, inconsistency is universally abhorred
% Hence the first sense of universality. 
% This is the first sense in which . 
This is the first 
    sense in which minimizing inconsistency is a \emph{universal} objective. 
As we shall soon see, 
    % is the first sense in which it is universal,
this shared scorn for inconsistency makes the concept
    particularly useful.
% , as will 
        % see in \cref{part:univ-objective}.
    % soon see.
% This is the first sense in which wanting to minimize inconsistency is universal: everyone thinks it is a good idea. 
}%
% this makes the concept particularly useful.


We have already seen (in \cref{part:univ-model}) that PDGs capture a wide variety of epistemic representations used in the literature.
\newmaterial{%
% Because PDGs can express so many different states of mind,
It follows that our formal definition of a PDG's degree of \emph{inconsistency} \cref{eqn:inconsistency-defn} also applies in a wide variety of contexts;
this is the second sense in which minimizing inconsistency is universal.

The technical material in \cref{part:univ-objective} demonstrates that
% This second sense is bolstered by the third one: 
many computations of interest---%
especially those that form the basis of modern machine learning and AI systems---%
can be viewed as approaches to inconsistency minimization.


Finally, it is universal in the sense that measuring inconsistency does not require making arbitrary choices---i.e., the opposite of ad hoc. Every aspect of the formula is well-motivated, all parameters have clear well-defended meanings, and the functional form of the 


These four notions of universality, while separable in principle, are in this case entangled aspects of a single concept:
% \begin{quotation}
minimizing inconsistency is always an appropriate thing to do. 
% \end{quotation} 
The fact that inconsistency is a shared enemy (meaning 1) explains why it is so often used in practice (meaning 3) and possible only because it applies to so many contexts (meaning 2). 

The fact that it makes sense in so many contexts (meaning 2) is precisely what enables it to be used in practice  (meaning 3).
Conversely, the fact that current practice can largely be viewed as inconsistency resolution (meaning 3) can be viewed as an empirical argument bolstering the (already strong) case that inconsistency is universally reviled (meaning 1). 

% By viewing identity of the value holder as part of the context, the fact that inconsistency makes sense in so many contexts (meaning 2) .

To summarize, it is already well appreciated that inconsistency is problematic (and so minimizing it is universal objective, in sense 1). We saw in \cref{part:univ-model} that PDGs can capture seemingly every standard representation of mental state. 
And now we proceed with the meat of the chapter: we will show that the overwhelmingly agreed-upon practices of building artificial intelligence can largely be cast as inconsistency minimization (the third sense in which inconsistency is a universal objective). 
This simultaneously (a) augments our theoretical arguments that inconsistency is problematic by showing that in practice people build intelligent systems by minimizing inconsistency, and (b) provides a principled way of choosing objectives even in situations when there is not an overwhelmingly popular standard to fall back upon.

% Thus, it can be viewed as an empirical argument that, in practice, 
% people do want to reduce inconsistency in essentially every context (meaning 1). 
}%
% Here, we argue that minimizing inconsistency is a \emph{universal objective}: it applies in almost any context, and reducing inconsistency is essentially always an appropriate thing to do.


\commentout{
In this part, we will 
% see how much of what we 
make this argument by showing that many computations of interest---%
especially those that form the basis of modern machine learning and AI systems---%
can be viewed as approaches to inconsistency minimization.
Thus, it can be viewed as an empirical argument, showing that, in practice, 
people do want to reduce inconsistency in essentially every context. 
}%
