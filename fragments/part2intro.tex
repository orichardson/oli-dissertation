% INTRODUCTION TO PART II
% The overarching objective of this work is to minimize inconsistency. This notion of minimizing inconsistency can be viewed as universal in four closely related ways:
% 
% Inconsistency is widely recognized as problematic, with many scholars arguing against the holding of conflicting beliefs. This shared view of inconsistency as undesirable makes the concept particularly useful to focus on.
% The formal definition of inconsistency developed in Part I applies across a wide variety of epistemic representations used in the literature. This broad applicability is a key aspect of the universality of inconsistency minimization.
% Much of the technical work in Part II shows that common computations in modern machine learning and AI systems can be understood as approaches to minimizing inconsistency. This suggests that inconsistency minimization is a central, if often implicit, goal in current AI practice.
% The measurement of inconsistency proposed here is principled and well-motivated, without relying on arbitrary choices. This objectivity is another dimension of the universality of this concept.
% 
% These four notions of universality - inconsistency as a shared enemy, the broad applicability of the formal framework, the centrality of inconsistency minimization in AI practice, and the principled nature of the inconsistency measure - are in fact entangled aspects of a single overarching concept. The fact that inconsistency is so widely reviled (the first sense of universality) is precisely what enables it to be used so broadly in practice (the second and third senses). Conversely, the pervasiveness of inconsistency minimization in current AI (the third sense) can be seen as empirical support for the view that inconsistency is universally problematic (the first sense). In this way, the different facets of universality are closely intertwined.
% Ultimately, minimizing inconsistency emerges as an appropriate and useful objective that underpins much of the work in this dissertation.










\newmaterial{%
% So as not to keep anyone in suspense, the \emph{universal objective} after which this part is named 
% So as not to keep anyone in suspense, this \emph{universal objective}
% The \emph{universal objective} that is the subject of \cref{part:univ-objective} is to minimize inconsistency.  
The thesis of \cref{part:univ-objective} is that minimizing inconsistency is a \emph{universal objective}.  
    % \newmaterial{%
% In what sense 
It may seem plausible that this is a good idea, but what justifies calling it ``universal''?
To a discerning eye, our usage of the word \emph{universal} 
% captures  four closely related concepts. 
    can be read in several different ways. 
Yet these different readings are closely related and reinforce one another; we argue that they are different aspects of a single coherent concept.
%

\begin{enumerate}
\item
From the very beginning (\cref{sec:defend-inconsist,sec:pdg-intro-examples}), we have argued that inconsistency is worth representing, 
    and less problematic than had been previously thought.
But that is a long way from an endorsement of holding inconsistent beliefs.
%
It seems everyone agrees that conflicting beliefs is worthy of derision,
    a position which has also been argued by scholars from many different angles 
\citep{descartes,sep-dutch-book,finocchiaro1981fallacies,%
    %priest1996paraconsistent,
    why-be-consistent-sugden}. 
    % In fact, inconsistency is universally abhorred
% Hence the first sense of universality. 
% This is the first sense in which . 
This is the first and most important
    sense in which minimizing inconsistency is a \emph{universal} objective. 
As we shall soon see, 
    % is the first sense in which it is universal,
this shared (``universally held'') scorn for inconsistency makes the concept
    particularly useful.
% , as will
        % see in \cref{part:univ-objective}.
    % soon see.
% This is the first sense in which wanting to minimize inconsistency is universal: everyone thinks it is a good idea. 
% this makes the concept particularly useful.


\item
We saw in \cref{part:univ-model} that PDGs capture an enormous variety of epistemic representations in use.
% Because PDGs can express so many different states of mind,
It follows that our formal definition of \emph{inconsistency} \cref{eqn:inconsistency-defn} also applies in a wide variety of contexts
    (making the notion ``universally applicable'').
% this is the second sense in which minimizing inconsistency is universal.
% this broad applicability is a second aspect of the universality of 

\item
The technical results in the coming part (\cref{part:univ-objective}) demonstrate that
% This second sense is bolstered by the third one: 
many standard measures of discrepancy, loss, and conflict, can be viewed as measuring the inconsistency of the appropriate PDG (\cref{chap:one-true-loss}), 
    so inconsistency can be viewed as a ``universal'' loss function.
Meanwhile, many computations of interest---%
especially those that form the basis of decision making and modern machine learning and AI systems---%
can be viewed as approaches to inconsistency minimization
    (\cref{chap:LIR}),
    and so minimizing inconsistency is a ``universal objective'' in that the most celebrated algorithms in AI can be written as approaches to inconsistency minimization.

\item 
The way we measure inconsistency is natural and does not require arbitrary choices. Every aspect of the formula is well-motivated and can be defended from first principles (making it a ``universal construction''). 
\end{enumerate} 


These four notions of universality, while separable in principle, are in this case entangled aspects of a single overarching concept.
% :
% \begin{quotation}
% minimizing inconsistency is always an appropriate thing to do. 
% \end{quotation} 
% The fact that inconsistency is so broadly applicable (the second sense of universality) is precisely why it is possible for it to be so widely reviled (the first sense), and for it to explain so many different popular learning and inference procedures used in practice (the third sense). 
The fact that inconsistency is so broadly applicable (sense 2) is precisely why it is possible for it to be so widely reviled (sense 1), and for it to explain so many different popular learning and inference procedures used in practice (sense 3). 
% The fact that inconsistency is so widely reviled (the first sense of universality) explains why it is so often used in practice (the third sense);
The fact that inconsistency is so widely reviled (sense 1) explains why it is so often used in practice (sense 3);
 % and possible only because it applies to so many contexts (meaning 2). 
% conversely, the fact that current practice can be viewed as inconsistency resolution 
conversely, the fact that so many minimization objectives in AI can be fruitfully represented as inconsistencies
% while the converse
(sense 3)
can be taken as as an empirical argument bolstering the case that, in practice, everyone wants to minimize inconsistency (sense 1)
\unskip. 
% The fact that it makes sense in so many contexts (meaning 2) is precisely 
%     why it makes sense for it to be universally reviled (sense 2) what enables it to be used in practice (meaning 3).
The fact that the construction is natural (universal in the fourth sense) makes it more plausible that it might be objectively a good idea to optimize, and explains why it applies so universally (in the second sense) and captures so many different loss functions (is universal in the third sense)---and the converse bolsters our claim that the inconsistency is a natural construction. 

% By viewing identity of the value holder as part of the context, the fact that inconsistency makes sense in so many contexts (meaning 2) .

% To summarize, all four notions are different entangled aspects of a single concept: it is 
%  it is already universally appreciated that inconsistency is problematic; 
 % (and so minimizing it is universal objective, in sense 1). 
% We saw in \cref{part:univ-model} that PDGs can capture seemingly every standard representation of mental state, an. 
We now turn to the primary task of \cref{part:univ-objective}: establishing that inconsistency is universal in the third sense: that the overwhelmingly agreed-upon practices of developing and working with artificial intelligence can be fruitfully cast as inconsistency resolution. 
In doing so, we will start to see how inconsistency also provides a principled way of choosing an objective (universal in senses 1 and 4) in any context (sense 2).
% This simultaneously (a) augments our theoretical arguments that inconsistency is problematic by showing that in practice people build intelligent systems by minimizing inconsistency, and (b) provides a principled way of choosing objectives even in situations when there is not an overwhelmingly popular standard to fall back upon.
Ultimately, minimizing inconsistency emerges as a universally appropriate, applicable, useful, and natural objective---or, for short, a \emph{universal} one. 


% Thus, it can be viewed as an empirical argument that, in practice, 
% people do want to reduce inconsistency in essentially every context (meaning 1). 
}%
% Here, we argue that minimizing inconsistency is a \emph{universal objective}: it applies in almost any context, and reducing inconsistency is essentially always an appropriate thing to do.


\commentout{
In this part, we will 
% see how much of what we 
make this argument by showing that many computations of interest---%
especially those that form the basis of modern machine learning and AI systems---%
can be viewed as approaches to inconsistency minimization.
Thus, it can be viewed as an empirical argument, showing that, in practice, 
people do want to reduce inconsistency in essentially every context. 
}%
