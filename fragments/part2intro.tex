\newmaterial{%
% We now argue that PDGs 

In \cref{sec:defend-inconsist}, we argued that inconsistency might be less problematic than it seems, and worth representing.  
But we have consistently clear that it is still undesirable.
% Being inconsistent is clearly bad, al
Indeed, inconsistency is universally reviled. 
As we will see, this makes the concept particularly useful.

We have already seen PDGs capture a wide variety of epistemic representations used in the litereature. 
Here, we argue that minimizing inconsistency is a \emph{universal objective}: it applies in almost any context, and reducing inconsistency is essentially always an appropriate thing to do.

In this part, we will 
% see how much of what we 
make this argument by showing that many computations of interest---%
especially those that form the basis of modern machine learning and AI systems---%
can be viewed as approaches to inconsistency minimization.
Thus, it can be viewed as an empirical argument, showing that, in practice, 
people do want to reduce inconsistency in essentially every context. 
}%
